\documentclass{beamer}
\usetheme{Antibes}

\usepackage{soul}			%For underlines

\title[\LaTeX{} Workshop]{Typesetting Linguistic Documents in \LaTeX}
%\subtitle[subtitle]{Uyghur vowel harmony}
\author[Saucic and Politzer-Ahles]{Mircea Saucic and Steve Politzer-Ahles}
\institute[KU]{
  Department of Linguistics\\
  University of Kansas
}
\date[October 2010]{7 October 2010}

\begin{document}

\begin{frame}[plain]
	\titlepage
\end{frame}

\begin{frame}{What is \LaTeX{}?}
	\begin{itemize}
	\pause
	\item A \emph{typesetting} language
	\pause
	\item Content-oriented
	\pause
	\item Extendable
	\pause
	\item Provides clear organization within a document
	\pause
	\item Not WYSIWYG
	\end{itemize}
\end{frame}

\begin{frame}{Basics}
	\begin{itemize}
	\item TeX file:
	\pause
		\begin{itemize}
		\item Preamble (header):
			\begin{itemize}
			\item \texttt{$\backslash$documentclass\{\}} (usually \texttt{article})
			\item Header information (metadata, etc.)
			\item Packages
			\pause
			\end{itemize}
		\item Document
			\begin{itemize}
			\item Environments (\texttt{$\backslash$begin\{document\}}, etc.)
			\item Commands (\texttt{$\backslash$emph\{\}}, etc.)
			\pause
			\end{itemize}
		\end{itemize}
	\item Build to PDF or other document format
	\end{itemize}
\end{frame}

\begin{frame}{Lists of examples}
	\begin{itemize}
	\pause
	\item Use package \texttt{gb4e} or \texttt{covington}
		\begin{itemize}
		\item (we'll use \texttt{gb4e} for this workshop)
		\end{itemize}
	\pause
	\item \texttt{$\backslash$begin\{exe\}} ... \texttt{$\backslash$end\{exe\}} creates the example environment
	\pause
	\item \texttt{$\backslash$ex\{} ... \texttt{\}} creates an example
	\pause
	\item \texttt{$\backslash$ex[*]\{} ... \texttt{\}} adds a grammaticality judgment
		\begin{itemize}
		\item (\texttt{*} can be replaced with \texttt{\#}, \texttt{?}, \texttt{\%}, etc.)
		\end{itemize}
	\pause
	\item \texttt{$\backslash$begin\{xlist\}} inside \texttt{$\backslash$ex\{\}} creates an embedded sub-list
	\end{itemize}
\end{frame}

\begin{frame}{Interlinear glosses}
	\begin{itemize}
	\pause
	\item Use package \texttt{gb4e} or \texttt{covington}
		\begin{itemize}
		\item (we'll use \texttt{gb4e} for this workshop)
		\end{itemize}
	\pause
	\item \texttt{$\backslash$gll} for the sentence and glosses (words separated by spaces)
		\begin{itemize}
		\item make sure there are the same number of words in each
		\item end each line with $\backslash\backslash$ (the newline character)
		\end{itemize}
	\pause
	\item \texttt{$\backslash$glt} for the free translation
	\end{itemize}
\end{frame}


\begin{frame}{IPA}
	\begin{itemize}
	\pause
	\item Use the \texttt{tipa} package
	\pause
	\item IPA inside the \texttt{$\backslash$textipa\{} ... \texttt{\}} command
	\pause
	\item See the tipa manual for list of IPA codes
	\end{itemize}
\end{frame}


\begin{frame}{Syntax trees}
	\begin{itemize}
	\pause
	\item Use the \texttt{qtree} package
	\pause
	\item Create a bracketed sentence
		\begin{itemize}
		\item Node labels are preceded with a period, e.g. \texttt{[.VP  V  N  ]}
		\item Create triangles using the command \texttt{$\backslash$qroof\{} ... \texttt{\}.XP}
		\end{itemize}
	\pause
	\item Draw arrows using package \texttt{tree-dvips}
	\end{itemize}
\end{frame}


\begin{frame}{Tables}
	\begin{itemize}
	\pause
	\item Use the \texttt{\{tabular\}} environment
	\pause
	\item \texttt{tabular} formatting is complicated to explain but easy to demonstrate. Check the example!
	\end{itemize}
\end{frame}

\begin{frame}{Footnotes and references}
	\begin{itemize}
	\pause
	\item Bigger or fancier bibliographies should use the BibTeX utility. We will just do simple ones today
		\begin{itemize}
		\item (but come ask us if you want to learn BibTeX!)
		\end{itemize}
	\pause
	\item The bibliography goes at the end of the document inside a \texttt{$\backslash$begin\{thebibliography\}} environment
	\pause
	\item Each reference is listed there inside after the command \texttt{$\backslash$bibitem\{\textit{key}\}}
		\begin{itemize}
		\item (replace ``key'' with a unique name for that reference, then you can refer to it in the article text)
		\end{itemize}
	\pause
	\item Put parenthetical citations within the text using using \texttt{$\backslash$cite\{\textit{key}\}}
	\pause
	\item You can make footnotes in the text with the \mbox{\texttt{$\backslash$footnote\{} ... \texttt{\}}} command
	\end{itemize}
\end{frame}


\begin{frame}{Where to find help}
	\begin{itemize}
	\pause
	\item \textul{latex-project.org} has many free general guides to LaTeX, including \textit{The (Not So) Short Introduction to LaTeX}
		\begin{itemize}\item (this is how I learned LaTeX!)\end{itemize}
	\pause
	\item \textul{CTAN.org} has specific guides for every package
	\pause
	\item Google \texttt{``latex for linguists''} to find many pages with info on how to use LaTeX for linguistic articles
	\pause
	\item All the slides and examples from today are available online at \textul{http://people.ku.edu/~sjpa/latex.html}
	\end{itemize}
\end{frame}

\end{document}