\documentclass{article}
\usepackage[margin=1in,
vmargin={0pt,1in},
includefoot,includehead]{geometry}
\usepackage{soul}

\begin{document}


Use [hbp] after the $\backslash$begin\{table\} command to tell \LaTeX{} where to try and put the table. It will try each of the locations you tell it to, in that
order, until it is able to put the table somewhere:

\begin{table}[hbp]
\begin{tabular}{c | l}
Command & Where it puts the table \\
\hline
\texttt{h} & right \textul{here} (i.e., the current position in the text) \\
\texttt{b} & at the \textul{bottom} of the page \\
\texttt{t} & at the \textul{top} of the page \\
\texttt{p} & on its own \textul{page}
\end{tabular}

\end{table}

\ \\

Now we will describe how the above table was made. To allow the table to float in your text, the whole thing has to go in a \{table\} environment ($\backslash$begin\{table\} \ldots{} $\backslash$end\{table\}).

\ \\

Inside that, the table itself is in a \{tabular\} environment. The first line of the environment specifies how many columns there are and how they're alined, as
well as the borders between columns:
\begin{quote}
\texttt{$\backslash$begin\{tabular\}\{c | l\}}
\end{quote}
specifies that you will be creating a table with one center-aligned column, then a vertical border, then a left-aligned column.

\ \\

After this, you can create the header of the table. Write the content of each column, separated by \texttt{\&}. End with the newline command ($\backslash$$\backslash$) to begin the next row:
\begin{quote}
\texttt{Command \& Where it puts the table $\backslash$$\backslash$}
\end{quote}

\ \\

Add \texttt{$\backslash$hline} to create a horizontal line, separating the header from the body of the table. Then create each row of the table, lining up the
columns and separating them with \texttt{\&} just like you did for the header. The end result should look like:
\begin{quotation}
\ \\
\texttt{$\backslash$begin\{table\}[hbp]}\\
\indent\texttt{$\backslash$begin\{tabular\}\{c | l\}}\\
\indent\texttt{Command \& Where it puts the table $\backslash$$\backslash$}\\
\indent\texttt{$\backslash$hline}\\
\indent\texttt{h \& right here (i.e., the current position in the text) $\backslash$$\backslash$}\\
\indent\texttt{b \& at the bottom of the page $\backslash$$\backslash$}\\
\indent\texttt{t \& at the top of the page $\backslash$$\backslash$}\\
\indent\texttt{p \& on its own page}\\
\indent\texttt{$\backslash$end\{tabular\}}\\
\texttt{$\backslash$end\{table\}}
\end{quotation}

\ \\

You can also do more advanced things with tables, such as having headers that span more than one row (using $\backslash$multicolumn) and adding captions, as shown
below. Consult the \LaTeX{} manual for details.

\begin{table}[hbp]
\begin{center}
\begin{tabular}{c l | c l}
\multicolumn{2}{c}{[i]} & \multicolumn{2}{c}{[u]} \\
 & & & \\
Token no. & F$_2$ (Hz) & Token no. & F$_2$ (Hz) \\
\hline
1 & 1885 & 11 & 1593 \\
2 & 1901 & 12 & 1643 \\
3 & 1867 & 13 & 1710 \\
4 & 2119 & 14 & 1593 \\
5 & 2120 & 15 & 2060 \\
6 & 2327 & 16 & 1810 \\
7 & 1912 & 17 & 1410 \\
8 & 2350 & 18 & 1510 \\
9 & 2477 & 19 & 1960 \\
10 & 2277 & 20 & 2093 \\
\multicolumn{2}{c}{$mean=2124$} & \multicolumn{2}{c}{$mean=1738$}
\end{tabular}
\end{center}
\caption{Frequency of F$_2$ for [s] tokens produced before [i] or [u]}
\end{table}

\end{document}