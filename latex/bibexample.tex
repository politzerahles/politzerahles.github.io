\documentclass[12pt,letterpaper]{article}

\begin{document}


Dikker and colleagues \cite{Dikker}: did not investigate ELAN directly, but used MEG\footnote{MEG has magnets.} and found that the visual M100 (generated in occipital lobe, visual areas) was sensitive to the same manipulations that are known to modulate the ELAN. Specifically, they found an increased M100 for phrase structure violations involving salient closed-class morphology (e.g. *the discovery was in the reported) but not for comparable phrase structure violations that were not marked by closed-class morphology (e.g. *the discovery was report). Based on these findings, they argue that early (100-150ms) waves thought to be responsive to phrase structure are actually low-level sensory responses, and by extrapolation, that the ELAN may be one such response.

Pauker and Steinhauer \cite{Pauker}: argue that the ELAN is an artifact caused by context effects (differing baselines across violation and control conditions) in prior experiments. They test this with a balanced design that includes expected verbs (the man hoped to enjoy the meal with friends), expected nouns (the man cooked the meal to enjoy with friends), unexpected verbs (*the man cooked the enjoy…), and unexpected nouns (*the man hoped to meal…). Therefore, when averaging both unexpected (phrase structure violation) conditions together and averaging both control conditions together, the overall comparison involved identical target words and identical baselines. In this averaged comparison, no ELAN effect was observed for violations; the earliest affect to emerge was a sustained LAN beginning at about 400ms, inconsistent with the typical timing of ELAN.\footnote{The ELAN is usually observed at 100-200ms.} Furthermore, when performing separate comparisons of control and violation conditions within the verb and noun groups, they found that the two violations elicited reverse effects: unexpected verbs elicited an early negativity (although it did not have a left anterior distribution), while unexpected nouns elicited an early positivity. The authors interpret this as evidence that the early effects are spillover effects from the preceding context (to and the in this example) and not indices of violation detection. The earliest point where both violations elicited similar responses regardless of context was after 400ms.


\begin{thebibliography}{9}
\bibitem{Dikker} Dikker, S.; Rabagliati, H.; Pylkkänen, L. (2009a). ``Sensitivity to syntax in visual cortex''. \emph{Cognition} 110. 293-321.
\bibitem{Pauker} Pauker, E.; Steinhauer, K. (2010). ``A marker of automatic syntactic processing or a context-related effect? The ELAN revisited''. 23rd CUNY Conference on Human Sentence Processing.
\end{thebibliography}
\end{document}